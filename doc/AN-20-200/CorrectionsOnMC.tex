\section{Corrections on MC}

\subsection{Luminosity Scale factors}

The number of generated MC events listed on tables \ref{tab:BkgList2016}, \ref{tab:BkgList2017},
and \ref{tab:BkgList2018} are arbitrary. In order to compare it with data, the
former has to be rescaled to account for the amount of data collected each year
during the data taking period. The luminosity scale factor is computed as follows:

\begin{equation}
  L_{sf}=\frac{L_{\rm data}}{L_{\rm MC}} = \frac{L_{\rm data}\times\sigma_{\rm MC}}{n_{\rm MC}},
\label{eq:lumiSF}
\end{equation}

Where $L_{\rm data}$ is the luminosity per year, see table \ref{tab:LuminosityPerYear},
$n_{MC}$ the number of generated MC events as provided by the sum of weighted events
from the generator (\verb|genWeight| branch on \verb|NanoAOD|), and $\sigma_{\rm MC}$
is the cross section of the MC samples as provided by \verb|XSecAnalyzer|.

\subsection{Pileup Reweighting}

\begin{figure}[tph]
  \centering
        \subfigure[Standard model WZ production normalized pileup distributions for Run 2 years]{\includegraphics[width=.5\textwidth]{fig/ScaleFactors/MC_Pileup_Linear.pdf}}
        \subfigure[Pileup profile from Run 2 Data, Linear Scale]{\includegraphics[width=.5\textwidth]{fig/ScaleFactors/GoldenPileup_Linear.pdf}}
        \vfil
        \subfigure[Pileup profile from Run 2 Data, Logarithmic Scale]{\includegraphics[width=.5\textwidth]{fig/ScaleFactors/GoldenPileup_LogScale.pdf}}
        \subfigure[Pileup weight profiles for several background samples]{\includegraphics[width=.5\textwidth]{fig/ScaleFactors/PileupWeight_LogScale.pdf}}
  \caption{Pileup profile distributions for MC and Data}
  \label{fig:RunII_PileupProfiles}
\end{figure}

The number of reconstructed primary vertices in each collision during a period
of time generates a distribution commonly referred as pileup. Montecarlo simulations
are produced with an estimated pileup profile and a correction is needed
in order to match the distributions obtained in the collected data. 
MC Samples are reweighted with $69.2~mb$ as MinBias cross section.
Figure \ref{fig:RunII_PileupProfiles} shows the differences between the
pileup distributions for run 2 data and montecarlo
simulations for the predominant background process (standard model
WZ production). These differences are then
corrected by computing a pileup weight bin by bin that matches the normalized
distribution from montecarlo with the normalized distribution from data.
The normalization of each distribution is eachieved by scaling it by a factor
equivalent to the the inverse of its integral. Each MC event is then reweighted
with the appropiate factor based on the number of reconstructed primary vertices. 

\subsection{Lepton scale factors}

\begin{figure}[tph]
  \centering
  \subfigure[Electron Loose ID]{\includegraphics[width=.5\textwidth]{fig/ScaleFactors/2016Electron_LooseID.pdf}}
  \subfigure[Electron Tight ID]{\includegraphics[width=.5\textwidth]{fig/ScaleFactors/2016Electron_TightID.pdf}}
  \vfil
  \subfigure[Muon GlobalHighPtId]{\includegraphics[width=.5\textwidth]{fig/ScaleFactors/2016Muon_GlobalHighPtID.pdf}}
  \subfigure[Muon TrkHighPtId]{\includegraphics[width=.5\textwidth]{fig/ScaleFactors/2016Muon_TrkHighPtID.pdf}}
  \caption{Lepton ID Scale Factors 2016}
  \label{fig:leptonidsf_2016}
\end{figure}

\begin{figure}[tph]
  \centering
  \subfigure[Electron Loose ID]{\includegraphics[width=.5\textwidth]{fig/ScaleFactors/2017Electron_LooseID.pdf}}
  \subfigure[Electron Tight ID]{\includegraphics[width=.5\textwidth]{fig/ScaleFactors/2017Electron_TightID.pdf}}
  \vfil
  \subfigure[Muon GlobalHighPtId]{\includegraphics[width=.5\textwidth]{fig/ScaleFactors/2017Muon_GlobalHighPtID.pdf}}
  \subfigure[Muon TrkHighPtId]{\includegraphics[width=.5\textwidth]{fig/ScaleFactors/2017Muon_TrkHighPtID.pdf}}
  \caption{Lepton ID Scale Factors 2017}
  \label{fig:leptonidsf_2017}
\end{figure}

\begin{figure}[tph]
  \centering
  \subfigure[Electron Loose ID]{\includegraphics[width=.5\textwidth]{fig/ScaleFactors/2018Electron_LooseID.pdf}}
  \subfigure[Electron Tight ID]{\includegraphics[width=.5\textwidth]{fig/ScaleFactors/2018Electron_TightID.pdf}}
  \vfil
  \subfigure[Muon GlobalHighPtId]{\includegraphics[width=.5\textwidth]{fig/ScaleFactors/2018Muon_GlobalHighPtID.pdf}}
  \subfigure[Muon TrkHighPtId]{\includegraphics[width=.5\textwidth]{fig/ScaleFactors/2018Muon_TrkHighPtID.pdf}}
  \caption{Lepton ID Scale Factors 2018}
  \label{fig:leptonidsf_2018}
\end{figure}

Trigger and lepton identification efficiencies are different between data
and montecarlo. Scale factors are provided centrally by the Muon and EGamma POG
and show an $\eta-P_{t}$ dependence, see Figures \ref{fig:leptonidsf_2016},
\ref{fig:leptonidsf_2017}, and \ref{fig:leptonidsf_2018}. Trigger and
identification systematics are evaluated by moving up and down the scale factors
by their uncertainties. These uncertainties are considered flat, independent
of the resonance mass, and uncorrelated in terms of lepton flavour.

