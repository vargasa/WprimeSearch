\section{Muon Momentum Resolution}

by travelling in Iron, High-Pt muons ($P_{t}>200GeV$) suffer from high energy
loss due to radiative processes. The transverse momentum measurement relays on
sagita ($s \propto 1/Pt$) which is associated with the number of hits found on
the Tracker and the Muon subsystems. As the number of hits is lower on the so called
tracker muons, i.e muons that were reconstructed from hits in the tracker and the
first station of the muon sistem, a study is needed to determine
the resolution of this Pt measurement for tracker-HighPt ID muons.

From MC simulations it is possible to access the true momenta of the muons and
a comparison with the reconstructed momentum can be given in terms of the
residual momentum:

\begin{equation}
  \frac{P_{GEN}}{P_{RECO}} - 1 = \frac{\dfrac{1}{P_{RECO}}-\dfrac{1}{P_{GEN}}}{\dfrac{1}{P_{RECO}}}
\end{equation}

The resolution can be extracted by fitting the event profile with a
Double-Sided Crystal Ball distribution wich consists of three diferent regions:
a gaussian center described by a mean $\mu$ and standard deviation $\sigma$ with two
exponential tails modelling the effects of radiative processes in the detector.
The two tails are controlled by four parameters, $n_L$ and $\alpha_L$ for the
left tail and $n_R$ and $\alpha_R$ for the right tail. $\alpha$ describes how
far away (as a factor of $sigma$) from the gaussian mean the exponential decay
intersects the guassian core, while $n$ controls the growth (decay) of the
left (right) exponential tail.


\begin{sidewaystable}[htb]
\begin{center}
  \caption{Montecarlo Samples used to study the Muon Momentum Resolution}
\footnotesize
\begin{tabular}{|l|l|l|}
\hline
Year & Sample & XSec [pb] \\ \hline
\hline
2016 & ZToMuMu\_NNPDF30\_13TeV-powheg\_M\_50\_120 & 2.116e+03\\
&ZToMuMu\_NNPDF30\_13TeV-powheg\_M\_120\_200 & 2.058e+01\\
&ZToMuMu\_NNPDF30\_13TeV-powheg\_M\_200\_400 & 2.890e+00\\
&ZToMuMu\_NNPDF30\_13TeV-powheg\_M\_400\_800 & 2.515e-01\\
&ZToMuMu\_NNPDF30\_13TeV-powheg\_M\_800\_1400 & 1.709e-02\\
&ZToMuMu\_NNPDF30\_13TeV-powheg\_M\_1400\_2300 & 1.370e-03\\
&ZToMuMu\_NNPDF30\_13TeV-powheg\_M\_2300\_3500 & 8.282e-05\\
&ZToMuMu\_NNPDF30\_13TeV-powheg\_M\_3500\_4500 & 3.414e-06\\
&ZToMuMu\_NNPDF30\_13TeV-powheg\_M\_4500\_6000 & 3.650e-07\\
&ZToMuMu\_NNPDF30\_13TeV-powheg\_M\_6000\_Inf & 2.526e-08\\
\hline
2017 & ZToMuMu\_NNPDF31\_TuneCP5\_13TeV-powheg-pythia8\_M\_50\_120 & 2.116e+03\\
&ZToMuMu\_NNPDF31\_TuneCP5\_13TeV-powheg-pythia8\_M\_120\_200 & 2.058e+01\\
&ZToMuMu\_NNPDF31\_TuneCP5\_13TeV-powheg-pythia8\_M\_200\_400 & 2.890e+00\\
&ZToMuMu\_NNPDF31\_TuneCP5\_13TeV-powheg-pythia8\_M\_400\_800 & 2.515e-01\\
&ZToMuMu\_NNPDF31\_TuneCP5\_13TeV-powheg-pythia8\_M\_800\_1400 & 1.709e-02\\
&ZToMuMu\_NNPDF31\_TuneCP5\_13TeV-powheg-pythia8\_M\_1400\_2300 & 1.370e-03\\
&ZToMuMu\_NNPDF31\_TuneCP5\_13TeV-powheg-pythia8\_M\_2300\_3500 & 8.282e-05\\
&ZToMuMu\_NNPDF31\_TuneCP5\_13TeV-powheg-pythia8\_M\_3500\_4500 & 3.414e-06\\
&ZToMuMu\_NNPDF31\_TuneCP5\_13TeV-powheg-pythia8\_M\_4500\_6000 & 3.650e-07\\
&ZToMuMu\_NNPDF31\_TuneCP5\_13TeV-powheg-pythia8\_M\_6000\_Inf & 2.526e-08\\
\hline
2018 & ZToMuMu\_NNPDF31\_TuneCP5\_13TeV-powheg-pythia8\_M\_50\_120 & 2.116e+03\\
&ZToMuMu\_NNPDF31\_TuneCP5\_13TeV-powheg-pythia8\_M\_120\_200 & 2.058e+01\\
&ZToMuMu\_NNPDF31\_TuneCP5\_13TeV-powheg-pythia8\_M\_200\_400 & 2.890e+00\\
&ZToMuMu\_NNPDF31\_TuneCP5\_13TeV-powheg-pythia8\_M\_400\_800 & 2.515e-01\\
&ZToMuMu\_NNPDF31\_TuneCP5\_13TeV-powheg-pythia8\_M\_800\_1400 & 1.709e-02\\
&ZToMuMu\_NNPDF31\_TuneCP5\_13TeV-powheg-pythia8\_M\_1400\_2300 & 1.370e-03\\
&ZToMuMu\_NNPDF31\_TuneCP5\_13TeV-powheg-pythia8\_M\_2300\_3500 & 8.282e-05\\
&ZToMuMu\_NNPDF31\_TuneCP5\_13TeV-powheg-pythia8\_M\_3500\_4500 & 3.414e-06\\
&ZToMuMu\_NNPDF31\_TuneCP5\_13TeV-powheg-pythia8\_M\_4500\_6000 & 3.650e-07\\
&ZToMuMu\_NNPDF31\_TuneCP5\_13TeV-powheg-pythia8\_M\_6000\_Inf & 2.526e-08\\
\hline
\end{tabular}
\label{tab:MomentumResolutionSamples}
\end{center}
\end{sidewaystable}

These studies were performed by simulating the standard model Z boson production
decaying to a pair of Muons ($Z\rightarrow\mu\mu$) the complete list of samples is
provided on table \ref{tab:MomentumResolutionSamples}
Z-Mass binned samples were used in order to increase the number of available
statistics in a wide range of transverse momentum, specially on the High-Pt sector.

\begin{figure}[tph]
  \centering
  \includegraphics[width=.5\textwidth]{fig/MomentumResolution/MomentumResolutionBins.png}
  \caption{Signal efficiency for individual and combined High Level Triggers for Run II}
  \label{fig:MomentumResolutionBins}
\end{figure}

A counting experiment following the trigger and muon selection described on
section %\ref{}%
collects the momentum resolution for each indivual Muon which is then classified
on different bins depending on its ID, either global or tracker High-Pt; the $\eta$
region in the Muon detector, i.e the barrel ($\lvert eta\lvert< 1.2$) or
the endcap ($1.2 < \lvert eta \lvert < 2.4$); and its transverse momentum range
(see figure \ref{fig:MomentumResolutionBins}).


\begin{figure}[tph]
  \centering
  \includegraphics[width=.7\textwidth]{fig/MomentumResolution/DSCB_Example.png}
  \caption{
    The Double-Sided Crystall Ball fit is shown as the blue solid line,
    the number of events as counted from MC samples are shown in black markers for
    the momentum resolution. The example shows the 2016 samples for muons labeled
    as globalHigh-Pt, seen in the barrel of the muon system. Different momentum
    ranges are shown, the
    top-left figure shows muons with momentum in the range $[52:72]~GeV$,
    the top-right figure $[72:100]~GeV$, bottom-left $[300:450]~GeV$,
    bottom-right $[450:800]~GeV$. The value for the width $\sigma$ for the DSCB
    distribution is shown on the horizontal axis and it shows evidence of how
    the resolution increases as the momentum range gets higher.
  }
  \label{fig:DSCB_Fit_Example}
\end{figure}

The distribution is then fitted by using a Double-Sided Crystall Ball (DSCB)
(\verb|RooCrystalBall| in \verb|RooFit|) which is a distribution defined
by parts, with a range-limited gaussian form around its mean with two
exponential tails which help model the nature of radiative processes occurring in
the detector. The width of the gaussian portion of the DSCB provides the
resolution (as a percentage) while the mean accounts for any bias present on the
momentum measurements. An example of the fit is shown in \ref{fig:DSCB_Fit_Example}

% Events with low statistics in bins are discarded

\subsection{Z-Mass Resolution}

\begin{sidewaystable}[htb]
\begin{center}
  \caption{Montecarlo Samples for Z Mass Resolution studies}
\footnotesize
\begin{tabular}{|l|l|l|}
\hline
Year & Sample & XSec [pb] \\ \hline
\hline
2016 & DYJetsToMuMu\_M-50\_Zpt-150toInf\_TuneCP5\_13TeV-madgraphMLM\_pdfwgt\_F-pythia8 & 6.178e+00\\
\hline
2017 & DYJetsToMuMu\_M-50\_Zpt-150toInf\_TuneCP5\_13TeV-madgraphMLM\_pdfwgt\_F-pythia8 & 6.182e+00\\
\hline
2018 & DYJetsToMuMu\_M-50\_Zpt-150toInf\_TuneCP5\_13TeV-madgraphMLM\_pdfwgt\_F-pythia8 & 6.178e+00\\
\hline
\end{tabular}
\label{tab:ZMassResolutionSamples}
\end{center}
\end{sidewaystable}

The dimuon mass can be used to validate simulation in data, where the true momentum
of the muon is unknown. For this, a Data/MC comparisson is done over the
distributions for the reconstructed dimuon mass. The distributions are built
from the reconstructed mass as a function of the muon transverse momentum and
obtained for each individual muon. The mass resolution distribution is then fitted with
a Breit-Weigner convolution of the Double-Sided crystal ball, from this fit the
standard deviation of the gaussian portion of the DSCB equates the dimuon
mass resolution.

